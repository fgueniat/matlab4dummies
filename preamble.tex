
\documentclass[9pt]{report}
%
\setcounter{secnumdepth}{5}

\usepackage{listings}
\usepackage{color}
\usepackage{xspace}
\usepackage{graphics}
\usepackage{graphicx,subfigure}
\usepackage[utf8]{inputenc}
\usepackage{ifthen}
\usepackage{algorithm}
\usepackage{algorithmic}
\usepackage{a4wide,amssymb,amsbsy,amsmath}
\usepackage[many]{tcolorbox}
\usepackage{tikz}
	\usetikzlibrary{calc}
	\usetikzlibrary{shapes,shadows,arrows,positioning}
	\usetikzlibrary{decorations.pathreplacing,decorations.pathmorphing}
\lstset{escapeinside={<@}{@>}}
\tolerance=1000
%% The lineno packages adds line numbers. Start line numbering with
%% \begin{linenumbers}, end it with \end{linenumbers}. Or switch it on
%% for the whole article with \linenumbers after \end{frontmatter}.
\begin{document}

%% \usepackage{lineno}


\renewcommand{\thesection}{\Roman{section}}
\renewcommand{\thesubsection}{{\color{gray}\thesection~}\alph{subsection})}
\renewcommand{\thesubsubsection}{{\color{gray}\thesubsection~}\roman{subsubsection}}
\renewcommand{\theparagraph}{{\color{gray}\thesubsubsection~}\arabic{paragraph}}

\newcommand{\comments}[1]{{\scriptsize\color{gray} #1}} % pour les commentaires

\DeclareRobustCommand{\matlab}{\texttt{MatLab}\xspace}
\DeclareRobustCommand{\mcode}[1]{\texttt{#1}\xspace}


%%%%%%%%%%%%%%%%%%%%%%%
% classique tools
\newcommand{\gras}[1]{\boldsymbol{#1}}
\newcommand{\mypar}[1]{\left(#1\right)}
\newcommand{\mya}[1]{\left\{#1\right\}}
\newcommand{\norme}[1]{\left\Vert #1\right\Vert_2}
\newcommand{\monabs}[1]{\left| #1\right|}


%%%%%%%%%% notations communes
\newcommand{\Ephaz}{\mathcal{D}}%espace des phases
\newcommand{\Eobs}{\Omega}%espace des observables

\newcommand{\Np}{n_p} % dim espace des phases
\newcommand{\Nobs}{n_r} % dim espace des observables
\newcommand{\Ns}{n_s} % dim sensors
\newcommand{\Nsnap}{N} % nombre de snapshots

%\newcommand{\flot}{\phi} % flot dynamique
\newcommand{\fdyn}{\mathfrak{f}} % fonction dynamique

%\newcommand{\velo}{\gras{u}} % champ de vitesse
\newcommand{\obs}{\gras{u}} % observable
\newcommand{\statei}{x} % comp de l'espace des phases
\newcommand{\state}{\gras{\statei}} % state de l'espace des phases
\newcommand{\freq}{\nu}
%%%%%%%%%% DMD notations
\newcommand{\Opev}{A} % Opérateur linéaire d'évolution

\newcommand{\Om}{\mathcal{K}} % matrice de Kalman
% observabilité notations
\newcommand{\dmdo}{\sigma} % DMD observability
\newcommand{\Pd}{\mathcal{S}} % Vecteur DMD observabilité = pertinence 
\newcommand{\chronos}{\xi}

% Define block styles
\newcommand{\todo}[1]{{ \center \LARGE\color{red}  [[TODO #1 ]] \\ }}    
\newcommand{\todoimage}[1]{\todo{ ** image #1 **}} 
   




\newtcolorbox{mytipbox}{%
  enhanced,
  boxsep=3pt,
  arc=1.25ex,
  colback=blue!5!white,
  colframe=blue!75!black,
  boxrule=3pt,
  leftrule=18pt,
  overlay unbroken and first ={%
    \node[rotate=90,
          minimum width=1cm,
          anchor=south,
          font=\Large\sffamily\bfseries,
          yshift=-18pt,
          white]
    at (frame.west) {Pro Tip};
  }
}
\newcommand{\tipbox}[1]{  { \begin{mytipbox} #1 \end{mytipbox} }  }   



\newtcolorbox{mydefbox}{%
  enhanced,
  boxsep=3pt,
  arc=1.25ex,
  colback=green!5!white,
  colframe=green!75!black,
  boxrule=3pt,
  leftrule=18pt,
  overlay unbroken and first ={%
    \node[rotate=90,
          minimum width=1cm,
          anchor=south,
          font=\Large\sffamily\bfseries,
          yshift=-18pt,
          white]
    at (frame.west) {Definition};
  }
}
\newcommand{\defbox}[2]{  { \begin{mydefbox} {\color{green!55!black}\textsc{#1}:} #2 \end{mydefbox} }  }   
\newtcolorbox{myhelpbox}{%
  enhanced,
  boxsep=3pt,
  arc=1.25ex,
  colback=orange!5!white,
  colframe=orange!75!black,
  boxrule=3pt,
  leftrule=18pt,
  overlay unbroken and first ={%
    \node[rotate=90,
          minimum width=1cm,
          anchor=south,
          font=\Large\sffamily\bfseries,
          yshift=-18pt,
          white]
    at (frame.west) {Help};
  }
}
\newcommand{\helpbox}[1]{  { \begin{myhelpbox} {Go and look for {\color{orange!55!black}\textsc{#1}} in the help for details} \end{myhelpbox} }  }   


\definecolor{mygreen}{RGB}{28,172,0} % color values Red, Green, Blue
\definecolor{mylilas}{RGB}{170,55,241}
\lstset{language=Matlab,%
    %basicstyle=\color{red},
    breaklines=true,%
    morekeywords={matlab2tikz},
    keywordstyle=\color{blue},%
    morekeywords=[2]{1}, keywordstyle=[2]{\color{black}},
    identifierstyle=\color{black},%
    stringstyle=\color{mylilas},
    commentstyle=\color{mygreen},%
    showstringspaces=false,%without this there will be a symbol in the places where there is a space
    numbers=left,%
    numberstyle={\tiny \color{black}},% size of the numbers
    numbersep=9pt, % this defines how far the numbers are from the text
    emph=[1]{end},emphstyle=[1]\color{black}, %some words to emphasise
    %emph=[2]{word1,word2}, emphstyle=[2]{style},    
}



	
\title{Introduction to \matlab.}


\author{Florimond Gu{\'e}niat}

%\authorrunning{Short form of author list} % if too long for running head



\maketitle

