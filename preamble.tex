
\documentclass[a4paper,9pt]{report}
%
\setcounter{secnumdepth}{5}

\usepackage[utf8]{inputenc}


%\usepackage[left=2cm,top=2cm,right=2cm,bottom=2.5cm]{geometry}
\usepackage{fancyhdr}
\usepackage{lastpage}

\usepackage{listings}
\usepackage{xspace}

\usepackage{color}

\usepackage{graphics}
\usepackage{graphicx,subfigure}
\usepackage{tikz}
	\usetikzlibrary{calc}
	\usetikzlibrary{shapes,shadows,arrows,positioning}
	\usetikzlibrary{decorations.pathreplacing,decorations.pathmorphing}


\usepackage{ifthen}
\usepackage{algorithm}
\usepackage{algorithmic}
\usepackage{a4wide,amssymb,amsbsy,amsmath}

\usepackage[many]{tcolorbox}
\tolerance=1000
%% The lineno packages adds line numbers. Start line numbering with
%% \begin{linenumbers}, end it with \end{linenumbers}. Or switch it on
%% for the whole article with \linenumbers after \end{frontmatter}.



%%%%%%%%%%%%%%%%%%%%%%%%%%%%%%%%%%%%%%%%%%%%%%%%%%%%%%%%%%%%%%%%%%%%%
%%%%%%%%%%%%%%%%%%%%%%%%%%%%%%%%%%%%%%%%%%%%%%%%%%%%%%%%%%%%%%%%%%%%%
%%%%%%%%%%%%%%%%%%%%%%%%%%%%%%%%%%%%%%%%%%%%%%%%%%%%%%%%%%%%%%%%%%%%%
% page layout

\setlength{\headheight}{15pt}
\footskip=50pt



  \lhead{\nouppercase{\rightmark} (\nouppercase{\leftmark})}
  \chead{}
  \rhead{}
  \cfoot{}
  \rfoot{\thepage}
  \renewcommand{\headrulewidth}{0.8pt}
  %\renewcommand{\footrulewidth}{0.4pt}
	\renewcommand{\headrule}{\hbox to\headwidth{%
  \color{blue!65!black}\leaders\hrule height \headrulewidth\hfill}}

  \renewcommand{\chaptermark}[1]{
		  \capmystring{\markboth{#1}{}}
  	}

\def\companylogo{\includegraphics[width=3.5cm]{./fig/logo/bcu.jpg}}
\newcommand{\univlogo}[1]{\includegraphics[width=#1cm]{./fig/logo/bcu.jpg}}
\fancypagestyle{companypagestyle}
{
    \fancyhf{}
	\fancyhead[R]{\slshape\nouppercase{\rightmark}}
	\fancyhead[L]{\slshape\nouppercase{\leftmark}}
    \fancyfoot[L]
    {
        \parbox[b]{\dimexpr\linewidth-2.5cm\relax}
        {
            Page \thepage\ of \pageref{LastPage}\hfill ENG 4xxx\\
            {\color{blue!65!black}\rule{\dimexpr\linewidth\relax}{0.4pt}}\\
            www.bcu.ac.uk
        }
    }
    \fancyfoot[R]
    {
		\parbox[b]{2cm}{\univlogo{3.5}}%
    }
}

\pagestyle{companypagestyle}


\renewcommand{\thesection}{\Roman{section}}
\renewcommand{\thesubsection}{{\color{gray}\thesection~}\alph{subsection})}
\renewcommand{\thesubsubsection}{{\color{gray}\thesubsection~}\roman{subsubsection}}
\renewcommand{\theparagraph}{{\color{gray}\thesubsubsection~}\arabic{paragraph}}


\begin{document}


%%%%%%%%%%%%%%%%%%%%%%%%%%%%%%%%%%%%%%%%%%%%%%%%%%%%%%%%%%%%%%%%%%%%%%
% matlab style

\lstset{escapeinside={<@}{@>}}


\definecolor{mygreen}{RGB}{28,172,0} % color values Red, Green, Blue
\definecolor{mylilas}{RGB}{170,55,241}
\lstset{language=Matlab,%
    %basicstyle=\color{red},
    breaklines=true,%
    morekeywords={matlab2tikz},
    keywordstyle=\color{blue},%
    morekeywords=[2]{1}, keywordstyle=[2]{\color{black}},
    identifierstyle=\color{black},%
    stringstyle=\color{mylilas},
    commentstyle=\color{mygreen},%
    showstringspaces=false,%without this there will be a symbol in the places where there is a space
    numbers=left,%
    numberstyle={\tiny \color{black}},% size of the numbers
    numbersep=9pt, % this defines how far the numbers are from the text
    emph=[1]{end},emphstyle=[1]\color{black}, %some words to emphasise
    %emph=[2]{word1,word2}, emphstyle=[2]{style},    
}


\newcommand{\comments}[1]{{\scriptsize\color{gray} #1}} % pour les commentaires

\DeclareRobustCommand{\matlab}{\texttt{MatLab}\xspace}
\DeclareRobustCommand{\mcode}[1]{\texttt{#1}\xspace}


%%%%%%%%%%%%%%%%%%%%%%%
% classique tools
\newcommand{\gras}[1]{\boldsymbol{#1}}
\newcommand{\mypar}[1]{\left(#1\right)}
\newcommand{\mya}[1]{\left\{#1\right\}}
\newcommand{\norme}[1]{\left\Vert #1\right\Vert_2}
\newcommand{\monabs}[1]{\left| #1\right|}


%%%%%%%%%%%%%%%%%%%%%%%%%%%%%%%%%%%%%%%%%%%%%%%%%%%%%%%%%%%%%%%%%%%%%% 
% TODO
\newcommand{\todo}[1]{{ \center \LARGE\color{red}  [[TODO #1 ]] \\ }}    
\newcommand{\todoimage}[1]{\todo{ ** image #1 **}} 
   



%%%%%%%%%%%%%%%%%%%%%%%%%%%%%%%%%%%%%%%%%%%%%%%%%%%%%%%%%%%%%%%%%%%%%%%%%%%
% Boxes
\newtcolorbox{mytipbox}{%
  enhanced,
  boxsep=3pt,
  arc=1.25ex,
  colback=blue!5!white,
  colframe=blue!75!black,
  boxrule=3pt,
  leftrule=18pt,
  overlay unbroken and first ={%
    \node[rotate=90,
          minimum width=1cm,
          anchor=south,
          font=\Large\sffamily\bfseries,
          yshift=-18pt,
          white]
    at (frame.west) {Pro Tip};
  }
}
\newcommand{\tipbox}[1]{  { \begin{mytipbox} #1 \end{mytipbox} }  }   



\newtcolorbox{mydefbox}{%
  enhanced,
  boxsep=3pt,
  arc=1.25ex,
  colback=green!5!white,
  colframe=green!75!black,
  boxrule=3pt,
  leftrule=18pt,
  overlay unbroken and first ={%
    \node[rotate=90,
          minimum width=1cm,
          anchor=south,
          font=\Large\sffamily\bfseries,
          yshift=-18pt,
          white]
    at (frame.west) {Definition};
  }
}
\newcommand{\defbox}[2]{  { \begin{mydefbox} {\color{green!55!black}\textsc{#1}:} #2 \end{mydefbox} }  }   
\newtcolorbox{myhelpbox}{%
  enhanced,
  boxsep=3pt,
  arc=1.25ex,
  colback=orange!5!white,
  colframe=orange!75!black,
  boxrule=3pt,
  leftrule=18pt,
  overlay unbroken and first ={%
    \node[rotate=90,
          minimum width=1cm,
          anchor=south,
          font=\Large\sffamily\bfseries,
          yshift=-18pt,
          white]
    at (frame.west) {Help};
  }
}
\newcommand{\helpbox}[1]{  { \begin{myhelpbox} {Go and look for {\color{orange!55!black}\textsc{#1}} in the help for details} \end{myhelpbox} }  }   





%\title{Introduction to \matlab.}


%\author{Florimond Gu{\'e}niat}

%\authorrunning{Short form of author list} % if too long for running head


%----------------------------------------------------------------------------------------
%	TITLE PAGE
%----------------------------------------------------------------------------------------

\begin{titlepage} % Suppresses displaying the page number on the title page and the subsequent page counts as page 1
	
	\raggedleft % Right align the title page
	
	{\color{blue!65!black}\rule{2pt}{\textheight}} % Vertical line
	\hspace{0.05\textwidth} % Whitespace between the vertical line and title page text
	\parbox[b]{0.75\textwidth}{ % Paragraph box for holding the title page text, adjust the width to move the title page left or right on the page
		
		{\large Modeling Mathematics ENG4xxx }\\[4\baselineskip] % class
		{\Huge\bfseries Notes on \matlab}\\[2\baselineskip] % Title
		{\large\textit{example-driven}}\\[4\baselineskip] % Subtitle or further description
		{\Large\textsc{Florimond Gueniat}} % Author name, lower case for consistent small caps
		
		\vspace{0.5\textheight} % Whitespace between the title block and the publisher
		
		{\noindent \univlogo{7}}\\[\baselineskip] % Publisher and logo
	}

\end{titlepage}

%\maketitle

