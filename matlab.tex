
\documentclass[]{report}
%
\setcounter{secnumdepth}{5}

\usepackage{listings}
\usepackage{color}
\usepackage{xspace}
\usepackage{graphics}
\usepackage{graphicx,subfigure}
\usepackage[utf8]{inputenc}
\usepackage{ifthen}
\usepackage{algorithm}
\usepackage{algorithmic}
\usepackage{a4wide,amssymb,amsbsy,amsmath}
\usepackage[many]{tcolorbox}
\usepackage{tikz}
	\usetikzlibrary{calc}
	\usetikzlibrary{shapes,shadows,arrows,positioning}
	\usetikzlibrary{decorations.pathreplacing,decorations.pathmorphing}
\lstset{escapeinside={<@}{@>}}
\tolerance=1000
%% The lineno packages adds line numbers. Start line numbering with
%% \begin{linenumbers}, end it with \end{linenumbers}. Or switch it on
%% for the whole article with \linenumbers after \end{frontmatter}.
\begin{document}

%% \usepackage{lineno}


\renewcommand{\thesection}{\Roman{section}}
\renewcommand{\thesubsection}{{\color{gray}\thesection~}\alph{subsection})}
\renewcommand{\thesubsubsection}{{\color{gray}\thesubsection~}\roman{subsubsection}}
\renewcommand{\theparagraph}{{\color{gray}\thesubsubsection~}\arabic{paragraph}}

\newcommand{\comments}[1]{{\scriptsize\color{gray} #1}} % pour les commentaires

\DeclareRobustCommand{\matlab}{\texttt{MatLab}\xspace}
\DeclareRobustCommand{\mcode}[1]{\texttt{#1}\xspace}


%%%%%%%%%%%%%%%%%%%%%%%
% classique tools
\newcommand{\gras}[1]{\boldsymbol{#1}}
\newcommand{\mypar}[1]{\left(#1\right)}
\newcommand{\mya}[1]{\left\{#1\right\}}
\newcommand{\norme}[1]{\left\Vert #1\right\Vert_2}
\newcommand{\monabs}[1]{\left| #1\right|}


%%%%%%%%%% notations communes
\newcommand{\Ephaz}{\mathcal{D}}%espace des phases
\newcommand{\Eobs}{\Omega}%espace des observables

\newcommand{\Np}{n_p} % dim espace des phases
\newcommand{\Nobs}{n_r} % dim espace des observables
\newcommand{\Ns}{n_s} % dim sensors
\newcommand{\Nsnap}{N} % nombre de snapshots

%\newcommand{\flot}{\phi} % flot dynamique
\newcommand{\fdyn}{\mathfrak{f}} % fonction dynamique

%\newcommand{\velo}{\gras{u}} % champ de vitesse
\newcommand{\obs}{\gras{u}} % observable
\newcommand{\statei}{x} % comp de l'espace des phases
\newcommand{\state}{\gras{\statei}} % state de l'espace des phases
\newcommand{\freq}{\nu}
%%%%%%%%%% DMD notations
\newcommand{\Opev}{A} % Opérateur linéaire d'évolution

\newcommand{\Om}{\mathcal{K}} % matrice de Kalman
% observabilité notations
\newcommand{\dmdo}{\sigma} % DMD observability
\newcommand{\Pd}{\mathcal{S}} % Vecteur DMD observabilité = pertinence 
\newcommand{\chronos}{\xi}

% Define block styles
\newcommand{\todo}[1]{{ \center \LARGE\color{red}  [[TODO #1 ]] \\ }}    
\newcommand{\todoimage}[1]{\todo{ ** image #1 **}} 
    
\newtcolorbox[auto counter]{mytipbox}{
freelance,
colback=white,
frame code={},
interior titled code={
  \fill[rounded corners=8pt,blue!50]
    (title.south west) --
    (title.south) -- 
    ([yshift=20pt]title.south) --
    ([yshift=20pt,xshift=4cm]title.south) --
    ([xshift=4cm]title.south) --
    (title.south east) {[sharp corners] --
    ([yshift=-6pt]title.south east) -- 
    ([yshift=-6pt]title.south west) } -- cycle;
  \draw[rounded corners=8pt,gray,line width=1pt]
    (title.west|-frame.south west) --
    (title.south west) --
    (title.south) -- 
    ([yshift=20pt]title.south) --
    ([yshift=20pt,xshift=4cm]title.south) --
    ([xshift=4cm]title.south) --
    (title.south east) --
    (title.east|-frame.south east) --
    cycle;
  \node at ([xshift=2cm,yshift=4pt,anchor=south]title.south) 
    {\sffamily\Large ProTip~\thetcbcounter};  
  },
title={\mbox{}},
top=12pt,
fontupper=\sffamily\Large,
oversize=0.5cm,
before={\vskip24pt\par\noindent},
after={\par\vskip12pt}
}


\newcommand{\tipbox}[1]{  { \begin{mytipbox} #1 \end{mytipbox} }  }   

\newtcolorbox[auto counter]{mydefbox}{
freelance,
colback=white,
frame code={},
interior titled code={
  \fill[rounded corners=8pt,green!30]
    (title.south west) --
    (title.south) -- 
    ([yshift=20pt]title.south) --
    ([yshift=20pt,xshift=4cm]title.south) --
    ([xshift=4cm]title.south) --
    (title.south east) {[sharp corners] --
    ([yshift=-6pt]title.south east) -- 
    ([yshift=-6pt]title.south west) } -- cycle;
  \draw[rounded corners=8pt,gray,line width=1pt]
    (title.west|-frame.south west) --
    (title.south west) --
    (title.south) -- 
    ([yshift=20pt]title.south) --
    ([yshift=20pt,xshift=4cm]title.south) --
    ([xshift=4cm]title.south) --
    (title.south east) --
    (title.east|-frame.south east) --
    cycle;
  \node at ([xshift=2cm,yshift=4pt,anchor=south]title.south) 
    {\sffamily\Large Definition~\thetcbcounter};  
  },
title={\mbox{}},
top=12pt,
fontupper=\sffamily\Large,
oversize=0.5cm,
before={\vskip24pt\par\noindent},
after={\par\vskip12pt}
}


\newcommand{\defbox}[2]{  { \begin{mydefbox} {\color{green}#1:} #2 \end{mydefbox} }  }   


\definecolor{mygreen}{RGB}{28,172,0} % color values Red, Green, Blue
\definecolor{mylilas}{RGB}{170,55,241}
\lstset{language=Matlab,%
    %basicstyle=\color{red},
    breaklines=true,%
    morekeywords={matlab2tikz},
    keywordstyle=\color{blue},%
    morekeywords=[2]{1}, keywordstyle=[2]{\color{black}},
    identifierstyle=\color{black},%
    stringstyle=\color{mylilas},
    commentstyle=\color{mygreen},%
    showstringspaces=false,%without this there will be a symbol in the places where there is a space
    numbers=left,%
    numberstyle={\tiny \color{black}},% size of the numbers
    numbersep=9pt, % this defines how far the numbers are from the text
    emph=[1]{end},emphstyle=[1]\color{black}, %some words to emphasise
    %emph=[2]{word1,word2}, emphstyle=[2]{style},    
}



	
\title{Introduction to \matlab.}


\author{Florimond Gu{\'e}niat}

%\authorrunning{Short form of author list} % if too long for running head



\maketitle















%%%%%%%%%%%%%%%%%%%%%%%%%%%%%%%%%%%%%%%%%%%%%%%%
%%%%%%%%%%%%%%%%%%%%%%%%%%%%%%%%%%%%%%%%%%%%%%%%
%%%%%%%%%%%%%%%%%%%%%%%%%%%%%%%%%%%%%%%%%%%%%%%%
\chapter{Quick tour \matlab}
%%%%%%%%%%%%%%%%%%%%%%%%%%%%%%%%%%%%%%%%%%%%%%%%
%%%%%%%%%%%%%%%%%%%%%%%%%%%%%%%%%%%%%%%%%%%%%%%%
\section{Legal stuff}
\matlab is a registered trademark of MathWorks, Inc.

%%%%%%%%%%%%%%%%%%%%%%%%%%%%%%%%%%%%%%%%%%%%%%%%
%%%%%%%%%%%%%%%%%%%%%%%%%%%%%%%%%%%%%%%%%%%%%%%%
\section{What is the use of \matlab ?}
\matlab~(for MATrix LABoratory) aims at delivering quickly some \emph{relatively} inexpensive computations.
It shines, as expected from the name, when it involves linear algebra, i.e., operations on matrices.

The main advantage of \matlab compared to language like C/C++ or Fortran are:
\begin{itemize}
	\item No compilation
	\item The prompt
	\item Simplicity
	\item portability
	\item Built-in functions:
		\begin{itemize}
			\item integration
			\item visualization
			\item tool-box
		\end{itemize}
\end{itemize}

The negative points are mostly:
\begin{itemize}
	\item sub performance
	\item not open source
	\item {\color{red} price}
\end{itemize}


%%%%%%%%%%%%%%%%%%%%%%%%%%%%%%%%%%%%%%%%%%%%%%%%
%%%%%%%%%%%%%%%%%%%%%%%%%%%%%%%%%%%%%%%%%%%%%%%%
\section{Equivalent of \matlab }

Octave and SciLab are almost identical to \matlab. A \matlab script would work on these two others open-source and free softwares.

Most of the tips can also be applied to python, especially when the packages scipy and numpy (for scientific and engineering computations) are used.


%%%%%%%%%%%%%%%%%%%%%%%%%%%%%%%%%%%%%%%%%%%%%%%%
%%%%%%%%%%%%%%%%%%%%%%%%%%%%%%%%%%%%%%%%%%%%%%%%
\section{Philosophical idea of these teachings}
The objective of these tutorial are to illustrate the ENG 4XXX courses as well at to help you to learn quickly how to numerically solve problems.
It will hence give you the first concepts behind \matlab, coding and problem solving.
The emphasis here is “learning by doing”. Therefore, try not to read these documents without a computer close-by.






%%%%%%%%%%%%%%%%%%%%%%%%%%%%%%%%%%%%%%%%%%%%%%%%
%%%%%%%%%%%%%%%%%%%%%%%%%%%%%%%%%%%%%%%%%%%%%%%%
\section{Hello World}

%%%%%%%%%%%%%%%%%%%%%%%%%%%%%%%%%%%%%%%%%%%%%%%%
\subsection{\matlab as a software}
%%%%%%%%%%%%%%%%%%%%%%%%%%%%%%%%%%%%%%%%%%%%%%%%
\subsubsection{Start \matlab}
Click on the icon, duh!
%%!!!!!!!!!!!!111
\subsubsection{Organization of the window}

You can find a few important sections:
\begin{itemize}
	\item the command windows \\
		\comments{This is the prompt}
	\item current folder \\
		\comments{It lists the files}
	\item the Workspace \\
		\comments{It gives details on the objects present in memory}
	\item the editor \\
		\comments{this is where you can write a script}
	\item the ribbon \\
		\comments{It gives access to properties, functions, etc. Similar in spirit to Words and Excel.}
\end{itemize}

\todoimage{opening windows of matlab}

%%%%%%%%%%%%%%%%%%%%%%%%%%%%%%%%%%%%%%%%%%%%%%%%
\subsection{How to print "hello world"}

Click on the Command Window, and type "hello world".
You will see:

\begin{lstlisting}
>> 'Hello World'
ans =
Hello World
\end{lstlisting}

\medskip

\matlab~printed the 'Hello World', congratulations !

You can see also that the 'Hello World has been assignated to a variable named ans.

\defbox{ans}{Short for answer. The results of the command is always stored in the variable ans, if it is not assigned to an other variable. }


Now, try without the quotes, and you will see:
\begin{lstlisting}
>> Hello World
<@{\color{red}Undefined function or variable 'Hello'.}@>
\end{lstlisting}
plus some help.

Hello World is understood by \matlab as a function/variable and then an option for this function.
\matlab hence thinks that Hello is something that already exists, and it is not the case here. An error follow.

The main reason behind that is that we want to print a string.

\defbox{String}{Chain of character. It has to be between quotes: 'some text' or double quotes: "some text".}


Try now to add a semi colon at the end of the line:
\begin{lstlisting}
>> 'Hello World';
>>
\end{lstlisting}

Nothing is printed in the prompt.

\defbox{Prompt}{The $>>$ sign. Once enter is hit, \matlab will interpret the line}

One of the most important tip to remember: \matlab will always print the result of a line if it does not have a ";" at the end of the line.

\tipbox{
Do not forget  the ";" at the end of lines. 
}






\section{\matlab as a calculator}
\subsection{Algebra}

You can use \matlab as a calculator. 
Click on the prompt: 

\begin{lstlisting}
>> 4+3
ans =
7
>> 4*3
ans =
12
\end{lstlisting}

As expected, \matlab respects the BODMAS (Brackets, Order, Division/Multiplication, Addition/Subtraction). 
Try a few operations !
\begin{lstlisting}
>> (4+3)*2
ans =
14
>> 4+3*2
ans =
10
\end{lstlisting}

\subsection{Variables}
ans can be used to store a result, but it will be overwritten every time a command is executed:

\begin{lstlisting}
>> 2+2
ans =
4
>> ans+2
ans =
6
>> ans+2
ans =
8
\end{lstlisting}

\subsection{Creation and re-assignement}

Variable can be easily created and assigned with the sign "=". 


\begin{lstlisting}
>> x = 2+2
x =
4
>> x+2
ans =
6
>> x*5
ans =
10
\end{lstlisting}

\defbox{variable}{a name that is associated with an value. Values can be results, functions or complex objects. They are usually assigned with the sign =}

%%!!!!!!!!!!!!111
\subsubsection{Naming convention}
A variable name can be anything, such as \mcode{goodnameforavariable} or  \mcode{GoodNameForAVariable}, or  \mcode{good\_name\_for\_a\_variable} .
However:
\begin{itemize}
	\item it cannot start with \_
	\item it cannot start with a number
	\item a few names are protected
\end{itemize}

\tipbox{Try to use clever name for variables, it will help to understand the code.  }

The choice of a name is important, for you a, \mcode{x} is good for an unknown, \mcode{s} if you expect its value to be a string, \mcode{v} if it is a vector... More complex names can be used, such as \mcode{x\_problem\_1}. Try to be consistent thorough the piece of code !

A few tips:

\begin{itemize}
	\item Use different names for different results
	\item Use a name that is meaningfull (e.g. \mcode{str\_name} if the variable is assigned with a chain of character that is a name)
	\item Consequently, avoid unecessary use of index (e.g. \mcode{result\_1, result\_2} etc.)
\end{itemize}

\tipbox{You can use the following name convention: \mcode{UpperCamelCase} for functions, \mbox{\mcode{CAPITALIZED\_WITH\_UNDERSCORES}} for constants, and \mbox{\mcode{lowercase\_separated\_by\_underscores}} for other variables.}


%%!!!!!!!!!!!!111
\subsubsection{Reassignement}
Updating a variable is handy: you might want to change the variable \mcode{year} from \mcode{2017} to \mcode{2018}.

You can easily update a variable, by reassigning a new value to it. It hence uses the sign "=".
For instance:
\begin{lstlisting}
>> x = 2+2
x = 
4
>> x = x + 5
x =
9
>> x = 0
x =
0
\end{lstlisting}

%%!!!!!!!!!!!!111
\subsubsection{Exercices}
\begin{enumerate}
	\item Create the variables \mcode{x,y,z} assigned with $1$, $2$ and $3$.
	\item Create the variable \mcode{sum\_xyz} that is the sum of \mcode{x,y} and \mcode{z}.
	\item Propose a name for a variable that is assigned as a value \mcode{'Birmingham'}
	\item Propose a name for a variable that is assigned as a value \mcode{'BCU'}
	\item Try to assign to the variable \mcode{year} the value \mcode{2017}, and then to \mcode{2018}!
	\item Try to assign to the variable \mcode{girlfriend\_name} the value \mcode{'Adilah'} (using the sign equal, pun totally intended), and to the variable \mcode{ex\_girlfriend\_name} the value \mcode{'Marie'}.
Then, reassign to the variable \mcode{girlfriend\_name} the value \mcode{'Kiara'}, and to the variable \mcode{ex\_girlfriend\_name} the value \mcode{'Adilah'}.

\end{enumerate}

\subsection{Entering multiple commands per line}
It is possible to enter multiple commands per line. 
Use commas "," or semicolons ";" for that ; the commas will \emph{not} suppress the outputs.

\tipbox{Try to avoid multiple commands per line, most of the time, it makes the code harder to read.}

\begin{lstlisting}
>> x = 2 ; y = 3 ; z = 4 ;
>> x = 2 ; y = 3 , z = 4 ;
y = 
3
>> x = 2 , y = 3 , z = 4 ,
x = 
2
y = 
3
z = 
5
\end{lstlisting}



\subsection{Basic arithmetic}


Basic arithmetic operators are pretty classic:

\begin{table}[h!]\caption{Arithmetic operators}
\center
\begin{tabular}{|l|c|c|}
	\hline
	operation & command & exemple \\
	\hline
	addition & + & 3+4 \\
	soustraction & - & 3-4 \\
 	multiplication & * & 3*4 \\
	division & / & 3/4 \\
	power & \^{} & 2 \^{} 4 \\
	\hline
\end{tabular}
\end{table}




%%%%%%%%%%%%%%%%%%%%%%%%%%%%%%%%%%%%%%%%%%%%%%%%
\subsection{functions}
%%!!!!!!!!!!!!111
\subsubsection{How to find a function or a command ?}

If you look for something, hit the help button.
For instance, if you want to look for the sine function:

\todoimage{help sine}


\tipbox{Use the help! It is \emph{very} useful and you will mostly find any function/tool/infos that you need.}

%%!!!!!!!!!!!!111
\subsubsection{Using a function}
Calling a function is relatively easy and intuitive. Let's take the sine function as an illustration. 

\begin{lstlisting}
>> sin(3.14)
ans =
0.0016
\end{lstlisting}

You ask \matlab to evaluate the function \mcode{sin} in $3.14\approx \pi$. 
For that, you just put the argument in parenthesis.

\tipbox{Trigonometric functions in \matlab are in radiant!}

Typical functions are available with an explicit name,see Tab.~\ref{tab-func}. Similarly, many constants are implemented in \matlab.

\begin{table}[h!]
	\caption{A few function names in \matlab. Many others are already implemented in \matlab.}
	\label{tab-func}
\center
\begin{tabular}{|l|c||l|c||l|c|}
	\hline
	Trigonometry & name & Stats & name & Misc. & name\\
	\hline
	sine & sin  &
		mean & mean &
			square root & sqrt \\
			
	cosine & cos &
		maximum & max &
			absolute value & abs \\

	exponential & exp &
		minimum & min &
			round up & ceil \\		

		
	natural logarithm & log &
		standard dev. & std &
			conjugate & conj \\
	
	\hline
\end{tabular}
\end{table}

\begin{table}[h!]
	\caption{A few useful constant names in \matlab.}
	\label{tab-func}
\center
\begin{tabular}{|l|c|}
	\hline
	$\pi$ & pi \\

	\hline
\end{tabular}
\end{table}

%%%%%%%%%%%%%%%%%%%%%%%%%%%%%%%%%%%%%%%%%%%%%%%%
\subsection{}


















%%%%%%%%%%%%%%%%%%%%%%%%%%%%%%%%%%%%%%%%%%%%%%%%
%%%%%%%%%%%%%%%%%%%%%%%%%%%%%%%%%%%%%%%%%%%%%%%%
%%%%%%%%%%%%%%%%%%%%%%%%%%%%%%%%%%%%%%%%%%%%%%%%
\chapter{Linear Algebra}

Linear algebra is the fundations of \matlab, and what makes it popular.

In particular, \matlab makes the manipulation of matrices and vectors very easy.
This section will show how to do:
\begin{itemize}
	\item operations involving arrays, or vectors
	\item operations involving matrices
	\item operations involving both arrays and matrices
\end{itemize}


\section{Vectors}
\subsection{Creation of a vector}
\subsubsection{Line vector}
Creating a vector is easy. You just put all the components between brackets.

Let's create Cartesian vectors in dimension two. $e_x = (1,0)$ and $e_y = (0,1)$:


\begin{lstlisting}
>> ex = [1 0]
ex =
1 0
>> ey = [0,1]
ey =
0 1
\end{lstlisting}
 
\tipbox{When creating a line vector, commas between components are unnecessary, but will help reading the code !}

\subsubsection{Column vector}
Creating a column vector is similar to creating a line vector, except that the element are separated with a semi-colon ";".


\begin{lstlisting}
>> ex = [1;0]
ex =
1
0
>> ey = [0;1]
ey =
0
1
\end{lstlisting}

\subsubsection{Transpose operator}
It is possible to change a column vector to a line vector, and reciprocaly, by using the transpose operator "'".
\begin{lstlisting}
>> ex = [1;0]
ex =
1
0
>> ex'
ans =
1 0
>> ey = [0,1]
ey =
0 1
>> ey'
ans =
0
1
\end{lstlisting}


\subsection{Access to the element of a vector}
Accessing the element of a vector is just calling the vector with specifying the desired element.

\begin{lstlisting}
>> x = [1 3 4 -2.5 8]
x = 
1 3 4 -2.5 8
>> x(1)
ans = 
1
>> x(3)
ans = 
4
>> x(4)
ans = 
-2.5
\end{lstlisting}




The last element of a vector can be called using the argument \mcode{end}:
You can also call the ith item from the end using \mcode{end-i}. 
\begin{lstlisting}
>> x = [1 3 4 -2.5 8]
x = 
1.000 3.000 4.000 -2.5 8.000
>> x(end)
ans = 
8
>> x(end-1)
ans = 
-2.5
>> x(end-2)
ans = 
4
\end{lstlisting}





\subsection{Basic operations on vectors}
\subsubsection{Addition/subtraction}
It is easy to add or subtract a given value to \emph{all} the components of a vector, using the signs "+" and "-".

\begin{lstlisting}
>> x = [1 3]
x = 
1 3
>> x + 4
ans =
5 7
>> y = x - 2
y =
-1 1
>> z = [5 10 -1 8] + 3
z = 
8 13 2 11
\end{lstlisting}


Vectors can be added, as long as their dimensions correspond:

\begin{lstlisting}
>> ex = [1 0]; ey = [0,1] ;
>> ex + ey
ans =
1 1
>> ex - ey
ans =
1 -1
\end{lstlisting}

Of course, if their dimensions do not correspond, \matlab will send back and error:

\begin{lstlisting}
>> x =[1,0] ; y = [1,0,0];
>> x+y
<@{\color{red}Matrix dimensions must agree.}@>
\end{lstlisting}




\tipbox{\matlab considers vectors as 1D matrices.}



\subsubsection{Multiplication}
It is easy to multiply or divide by a given value to \emph{all} the components of a vector, using the signs "*" and "/"

\begin{lstlisting}
>> x = [1 3]
x = 
1 3
>> x * 4
ans =
4 12
>> y = x / 2.
y =
0.5 1.5
>> z = 3 * [5 10 -1 8]
z = 
15 30 -3 24
\end{lstlisting}

What is multiplication for vectors ?

Two definitions can be proposed.

\paragraph{dot product}
The first definition is the dot product between two vectors.
\defbox{dot product}{The dot product (or inner product) of two vectors is the sum of the multiplication of their components. If $u=(u_i),v=(v_i)$ are $n$-dimensional vectors, $<u,v>= \sum\limits_{i=1}^n u_i \times v_i $.}


It can be done using the function \mcode{dot}.
\begin{lstlisting}
>> x =[1,0] ; y = [1,0];
>> dot(x,y)
ans = 
1
>> x(1) * y(1) + x(2) * y(2)
ans =
1
>> a =[0,0.5,2] ; b = [2,0,4];
>> dot(a,b)
ans = 
8
>> a(1) * b(1) + a(2) * b(2) + a(3) * b(3)    
and =
8
\end{lstlisting}


\paragraph{Element-wise multiplication}
Another definition could be the element wise multiplication.
It means that each element of a vector is multiply by the corresponding element of the other vector.
It is similar to the dot product \emph{except} for the sum.
The operator for that is ".*" (it is read "dot product", which is pretty stupid when you think about it!).

\begin{lstlisting}
>> x =[1,0] ; y = [1,0];
>> x.*y
ans = 
1 0
>> [x(1) * y(1), x(2) * y(2)]
ans =
1 0
>> a =[0,0.5,2] ; b = [2,0,4];
>> a.*b
ans = 
0 0 8
>> [a(1) * b(1), a(2) * b(2), a(3) * b(3)]
and =
0 0 8
\end{lstlisting}


\tipbox{In \matlab, using "." in front of an operator means that this operator will be applied element-wise (to each element of the vector/matrix).}


\todo{exercices}

\section{Matrices}
\matlab sees vectors a line matrices. Building a matrix is the equivalent of stacking lines.
For that, \matlab uses the semi colon sign ";".
The following lines are equivalent:
\begin{lstlisting}
>> m =[1,0 ; 0,1];
m = 
1 0
0 1
>> m =[1,0 ; 0 1];
m = 
1 0
0 1
>> m =[ [1,0] ; [0 1]];
m = 
1 0
0 1
\end{lstlisting}




\todo{exercices}
Try to create the vector $x = (1,2,3,4)$.

\bibliographystyle{ieeetr}
\bibliography{biblio}

\end{document}

%%
%% End of file `elsarticle-template-3a-num.tex'.
